% Copyright (C) 2023 Dr. Ramón L. Panadés Barrueta

% A minimal beamer template optimized for XeTeX

% This program is free software: you can redistribute it and/or modify it under the terms of the GNU General Public License as published by the Free Software Foundation, either version 3 of the License, or (at your option) any later version.

% This program is distributed in the hope that it will be useful, but WITHOUT ANY WARRANTY; without even the implied warranty of MERCHANTABILITY or FITNESS FOR A PARTICULAR PURPOSE. See the GNU General Public License for more details.

% You should have received a copy of the GNU General Public License along with this program. If not, see <https://www.gnu.org/licenses/>. 

\documentclass{beamer}

%%%%%%%%%%%%%%%%%%%%%%%%%%%%%%%%%%%%%%%%%%%%%%%%%%%%%%%%%%%%%%% 
% Packages
%%%%%%%%%%%%%%%%%%%%%%%%%%%%%%%%%%%%%%%%%%%%%%%%%%%%%%%%%%%%%%% 

\usepackage{fourier-otf}
\usepackage{fontspec}

%%%%%%%%%%%%%%%%%%%%%%%%%%%%%%%%%%%%%%%%%%%%%%%%%%%%%%%%%%%%%%% 
% Theme
%%%%%%%%%%%%%%%%%%%%%%%%%%%%%%%%%%%%%%%%%%%%%%%%%%%%%%%%%%%%%%% 

\usetheme{default}
\useinnertheme{metropolis}
\usecolortheme{default}
\beamertemplatenavigationsymbolsempty{}

% Font
\setmonofont{Fira Code}[
Contextuals=Alternate 
]

% Colors
\definecolor{bluish}{HTML}{267CB9}
\definecolor{reddish}{HTML}{b96326}
\definecolor{greenish}{HTML}{33b926}
\definecolor{secondary}{HTML}{1A5F8A}

% Customize the theme colors
\setbeamercolor{structure}{fg=bluish}
\setbeamercolor{alerted text}{fg=reddish}
\setbeamercolor{example text}{fg=greenish}
\setbeamercolor{footline}{fg=secondary}
\setbeamerfont{footline}{series=\bfseries}

% Bullets
\useinnertheme{circles}

% Blocks
\setbeamercolor{block body alerted}{bg=alerted text.fg!5}
\setbeamercolor{block title alerted}{bg=reddish, fg=white}

\setbeamercolor{block body}{bg=structure!5}
\setbeamercolor{block title}{bg=bluish, fg=white}

\setbeamercolor{block body example}{bg=greenish!5}
\setbeamercolor{block title example}{bg=greenish, fg=white}

\setbeamertemplate{blocks}[rounded][shadow]

%%%%%%%%%%%%%%%%%%%%%%%%%%%%%%%%%%%%%%%%%%%%%%%%%%%%%%%%%%%%%%% 
% Title page
%%%%%%%%%%%%%%%%%%%%%%%%%%%%%%%%%%%%%%%%%%%%%%%%%%%%%%%%%%%%%%% 
\title{Your title here}
\author{Your name here}
\institute{Your institution here}

\begin{document}

\setbeamertemplate{title page}[default]

\begin{frame}[noframenumbering]
  \titlepage{}
\end{frame}

\setbeamertemplate{footline}{
  \vspace{-4em}%
  \makebox[0.99\paperwidth][r]{
    \insertframenumber/\inserttotalframenumber{}
  }
}

\begin{frame}[noframenumbering]
  \frametitle{Outline}
  \tableofcontents
\end{frame}

%%%%%%%%%%%%%%%%%%%%%%%%%%%%%%%%%%%%%%%%%%%%%%%%%%%%%%%%%%%%%%% 
% Begin the fun!
%%%%%%%%%%%%%%%%%%%%%%%%%%%%%%%%%%%%%%%%%%%%%%%%%%%%%%%%%%%%%%% 

\section{Introduction}

\begin{frame}{Introduction}

  Some ideas to conquer the world:

  \begin{itemize}
  \item Learn math
  \item Think Julia
  \end{itemize}

  \begin{enumerate}
  \item Be nice
  \item and machiavelic
  \end{enumerate}

\end{frame}

\section{Some math}

\begin{frame}{Equations}
  \begin{block}{Simple}
    \(e^{i\pi}+1 =0\)
  \end{block}
  
  \begin{alertblock}{Dangerous}
    \(E^2={(pc)}^2+{(m_0c^2)}^2\)
  \end{alertblock}

  \begin{exampleblock}{Inspiring}
    \(\ell_s = \hbar c \sqrt{\alpha'}\)
  \end{exampleblock}
\end{frame}

\begin{frame}[plain,c,noframenumbering]
  \centering \Large
  \textbf{\color{bluish}{Thanks!}}
\end{frame}

\end{document}

% Local Variables:
% TeX-engine: xetex
% End:
